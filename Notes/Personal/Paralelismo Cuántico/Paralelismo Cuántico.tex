\documentclass[12pt,letterpaper]{article}

\usepackage[spanish,english]{babel}
\usepackage[utf8]{inputenc}
\usepackage{graphicx}
\usepackage{pdfpages}
\usepackage[hidelinks]{hyperref}
\usepackage{amsbsy,amsmath,amstext,amssymb,amsfonts,amsthm}
\usepackage[usenames,dvipsnames]{color}
\usepackage{listings}
\usepackage{verbatim}

%You can use \cofe[ABCD]m{alpha=[0,1]}{scale=[0:...]}{algle=[1,360]}{xoff}{yofff}

\newcounter{problem}
\newcommand\Problem{
    \stepcounter{problem}
    \textbf{Problema \theproblem\\[2.0pt]}
}

\newtheorem{theorem}{Teorema}[section]
\newtheorem{corollary}{Corolario}[theorem]
\newtheorem{lemma}[theorem]{Lema}
\newtheorem{definition}[theorem]{Definición}

\renewcommand\qedsymbol{$\blacksquare$}

\title{Paralelismo Cuántico}
\author{José Joaquín Zubieta Rico\\IBM Q \-- ITESM Workshop}
\date{\today}

\begin{document}

\maketitle

    Make possible evaluate a function $f(x)$ for each value of $x$ in a single step of the execution of a quantum algorithm.

    \section*{Deutsch-Jozsa Problem}

        We define a $f:{\left\{0,1\right\}}^{n}\rightarrow\left\{0,1\right\}$ function that can be one of tow types of functions:
        \begin{itemize}
            \item   A constant function in which case all the outputs of evaluating $f$ in some $x \in {\left\{0,1\right\}}^n$ will be zero.
            \item   a balanced function 
        \end{itemize}

    \section*{Deutsch-Jozsa Algorithm}

        Originally the Deutsch algorithm was developed with the intention of apply a single function over an ``array'' of possible states on a quantum computer. In the specific case of the Deutsch-Jozsa algorithm it is used for apply an oracle $(f)$ function over all possible values it can receive.
        For this case we will represent the module $2$ sum (a XOR logic gate) as a $\oplus$.

        \subsection*{Definition $f$ oracle function}

        \subsection*{Unitary operator for $f$}

            We will define an unitary operator $\^{U}_f$ that will map the state $|x\rangle\otimes|y\rangle$ into the state $|x\rangle\otimes|y+f(x)\rangle$, for our $f$ oracle function.

\end{document}

